\subsection{Group Actions}

Let $G$ be a group acting on a topological space $X$ via $G\times X\to X$,
$(g,x)\mapsto g\cdot x=g(x)$ such that
\begin{itemize}
\item each map $x\mapsto g\cdot x$ is continuous (as a fn $X\to X$)
\item $e\cdot x=x$
\item $g\cdot(h\cdot x)=(gh)\cdot x$ $\forall g,h\in G,x\in X,e=\text{id}_{G}$.
\end{itemize}
then $x\mapsto g\cdot x=g(x)$ is a homeomorphism.
\begin{defn}
The \emph{orbit} of a point $x$ is $G\cdot x=\{g\cdot x\mid g\in G\}$.
The \emph{orbit space} $X\backslash G$ (or $X/G$) is $X/\sim$ where
$x\sim y\Leftrightarrow y=g\cdot x$ for some $g\in G$.
\end{defn}

\begin{defn}
The free group $F(S)$ generated by $S$ is the set of all reduced
words in $S$ with multiplication defined by juxtaposition followed
by reduction.\end{defn}
\begin{prop}
If $G$ is a group generated by a subset $S$, and $F=F(S)$, then
$G$ is a quotient of $F$.\end{prop}
\begin{proof}
The inclusion $f:S\hookrightarrow G$ extends to a homomorphism $\bar{f}:F(S)\to G$.
Then $S=\{\text{set of generators}\}$ so $\bar{f}$ is onto. The
First Isomorphism Theorem then says $G\cong F/N$, $N=\text{ker}(\bar{f})=\text{normal subgroup of F}.$\end{proof}
\begin{defn}
A group $G$ is defined by a set of generators $S$ and a set of relators
$R$. $G\cong F(S)/N(R)$ where $N(R)=\{\text{smallest normal subgroup containing \ensuremath{R}}\}=\text{"normal closure of \ensuremath{R}"}.$
We write $G=\langle S\mid R\rangle$. A group is \emph{finitely presented}
if both sets are finite.\end{defn}
\begin{prop}
Let $G=\langle S\mid R\rangle$, $H$ any group, and $f:S\to H$ any
function. Then $f$ extends to a homomorphism $\tilde{f}:G\to H$
iff $f(x_{i})^{\epsilon_{1}}\cdots f(x_{n})^{\epsilon_{n}}=1$ for
every relator $x_{1}^{\epsilon_{1}}\cdots x_{n}^{\epsilon_{n}}$.
\end{prop}
\textit{\emph{Constrstructions}}

Direct product: $G_{1}\times G_{2}=\langle S_{1},S_{2}\mid R_{1},R_{2},[x,y]=1\forall x\in S_{1},\forall y\in S_{2}\rangle.$

Free product: $G_{1}*G_{2}=\langle S_{1},S_{2}\mid R_{1},R_{2}\rangle$

and Abelianisation: $G^{ab}=\langle S\mid R,[xy]=1\forall x,y\in S\rangle$

Note: Every finitely-generated Abelian group is isomorphic to a direct
product of cyclic groups $A\cong\mathbb{Z}^{n}\oplus\mathbb{Z}_{d_{1}}\oplus\cdots\oplus\mathbb{Z}_{d_{k}}.$
\begin{lem}
(Lebesgue number lemma) Let $(X,d)$ be a compact metrix space. Given
any open cover $\{U_{\alpha}\}$ of $X$, $\exists\delta>0$, the
Lebesgue number, such that any subset of $X$ with $\text{diam}<\delta$
is contained in one of the $U_{\alpha}$.
\end{lem}
We can use this to help prove Seifert-van Kampen.
\begin{thm}
(Seifert-van Kampen Theorem)

Let $X=U_{1}\cup U_{2}$ with $U_{1},U_{2},U_{0}=U_{1}\cap U_{2}$
are all open, non-empty and path connected. Choose a basepoint $x_{0}\in U_{0}$
for all fundamental groups that we will be looking at. We have the
inclusion maps

\begin{align*}
\xymatrix{ & U_{1}\ar[dr]_{j_{1}}\\
U_{0}\ar[ur]^{i_{1}}\ar[dr]_{i_{2}} &  & X\\
 & U_{2}\ar[ur]_{j_{2}}
}
\end{align*}


which induces the homomorphisms

\begin{align*}
\xymatrix{ & \pi_{1}(U_{1})\ar[dr]_{j_{1*}}\\
\pi_{1}(U_{0})\ar[ur]^{i_{1*}}\ar[dr]_{i_{2*}} &  & \pi_{1}(X)\\
 & \pi_{1}(U_{2})\ar[ur]_{j_{2*}}
}
\end{align*}


then, $\pi_{1}(X)$ is obtained from the free product $\pi_{1}(U_{1})*\pi_{1}(U_{2})$
by adding the ``amalgamation'' relations:

\begin{align*}
(i_{1*})(u) & =(i_{2*})(u)\forall u\in\pi_{1}(U_{0}).
\end{align*}


More explicitly: If $\pi_{1}(U_{1})=\langle S_{1}\mid R_{1}\rangle$,
$\pi_{1}(U_{2})=\langle S_{2}\mid R_{2}\rangle$, $\pi_{1}(U_{0})=\langle S_{0}\mid R_{0}\rangle$
then $\pi_{1}(X)=\langle S_{1},S_{2}\mid R_{1,}R_{2}(i_{1*})(u)=(i_{2*})(u)\forall u\in S_{0}\rangle$.

More precisely, the homomorphisms $(j_{1*}):\pi_{1}(U_{1})\to\pi_{1}(X),(j_{2*}):\pi_{1}(U_{2})\to\pi_{1}(X)$
extend to homomorphism $\pi_{1}(U_{1})*\pi_{1}(U_{2})\to\pi_{1}(X)$
with kernal generated by $\{(i_{1*})(u)(i_{2*})(u)^{-1}\mid u\in\pi_{1}(U_{0})\}$.
\end{thm}

