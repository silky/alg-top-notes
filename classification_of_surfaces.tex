\subsection{Classification of surfaces}

Given surfaces $M_{1}$ and $M_{2}$ we can form a new surface $M_{3}=M_{1}\#M_{2}$,
the connected sum, by removing from each the interior of ~disc, and
gluing together along the boundaries with a homeomorphism $f$.
\begin{fact}
Every closed connected surface is homeomorphic to one of $S^{2}$,
$M_{g}=T\#T\#\cdots\#T$ ($g$ copies of the torus $T=T^{2}$) or
$N_{h}=P\#P\#\cdots\#P$ (h copies of $P=\mathbb{R}P^{2})$.
\end{fact}
$M_{g}$ can also be understood in a planar model as the $4g$-gon
with edges identified in pairs, i.e. according to the word

\begin{align*}
[a_{1},b_{1}][a_{2},b_{2}]\cdots[a_{g},b_{g}].
\end{align*}


$N_{h}$ can be written as the $2h$-gon with edges identified according
to the word

\begin{align*}
a_{1}^{2}a_{2}^{2}\cdots a_{n}^{2}.
\end{align*}


Note: We can see that these surfaces are all different (not homeomorphic)
by considering the Abelianisations. We then have some corollaries.
\begin{cor}
No two $S^{2},M_{g},N_{h}$ are homeomorphic. In fact, two closed
surfaces are homeomorphic iff they have the same Abelian $\pi_{1}$,
iff they are homotopically-equivalent.
\end{cor}

\begin{cor}
A closed surface is simply connected iff it is homeomorphic to $S^{2}$.
\end{cor}

