\subsection{Homology theory}

Let's remind ourselves of the First Isomorphism Theorem
\begin{thm}
(First Isomorphism Theorem) Let $G$, $H$ be groups, and let $\phi:G\to H$
be a homomorphism, then\end{thm}
\begin{enumerate}
\item $\text{Img}(\phi)\cong G/\ker(\phi)$,
\item $\ker(\psi)$ is a normal subgroup of $G$,
\item $\text{Img}(\psi)$ is a subgroup of $H$,
\end{enumerate}
In particular if $\phi$ is onto (surjective) then $H\cong G/\ker(\psi)$.
\begin{defn}
The standard $p$-simplex is 

\begin{align*}
\Delta^{p} & =\{(t_{0},\dots,t_{p})\in\mathbb{R}^{p+1}\mid t_{i}\geqslant0,\sum_{i}t_{i}=1\}.
\end{align*}

\end{defn}

\begin{defn}
Let $X$ be a topological space. A singular $p$-simplex is a continuous
map $\sigma:\Delta^{p}\to X$.
\end{defn}

\begin{defn}
A $p$-chain is denoted $C_{p}(X)=\{\text{singular \ensuremath{p}-simplesx in \ensuremath{X}}\}=\text{free Abelian group generated by all singular \ensuremath{p}-simplexes in \ensuremath{X}.}$
So a $p$-chain is a formal sum

\begin{align*}
C_{p}(X)=\sum_{i}n_{i}\sigma_{i}
\end{align*}


where each $\sigma_{i}$ is a signular $p$-simplex in $X$ (and all
but finitely-many $n_{i}$ are $0$.)
\end{defn}

\begin{defn}
(Boundaries)

The $i$th face map for the standard $p$-simplex is the affine map

\begin{align*}
[e_{0},\dots,\hat{e}_{i},\dots,e_{p}] & :\Delta^{p-1}\to\Delta^{p}
\end{align*}


(where the hat means we \emph{omit} this vertex.) Let $\sigma^{(i)}:\Delta^{p-1}\to X$
be given by

\begin{align*}
\sigma^{(i)} & =\sigma\circ[e_{0},\dots,\hat{e}_{i},\dots,e_{p}]
\end{align*}


then defined the boundary of $\sigma$, $\partial\sigma$ as

\begin{align*}
\partial\sigma & =\sum_{i=0}^{p}(-1)^{i}\sigma^{(i)}
\end{align*}


and we have $\partial\sigma\in C_{p-1}(X)$. So extend by linearity
to get a homeomorphism $\partial=\partial_{p}:C_{p}(X)\to C_{p-1}(X)$.
Say $C_{p}(X)=0$ for $p<0$, and that $\partial\sigma=0$ for an
singular $0$-chain.\end{defn}
\begin{lem}
$\partial\circ\partial=0$. (Proof by linearity, doing the sums, separating
and noting that they cancel.)\end{lem}
\begin{defn}
Cycles and boundaries\end{defn}
\begin{itemize}
\item $\alpha\in C_{p}(X)$ is a $p$-cycle if $\partial\alpha=0$.
\item $\alpha$ is a $p$-boundary if $\alpha=\partial\beta$ for $\beta\in C_{p+1}(X)$
\item Let $Z_{p}=\{\text{\ensuremath{p}-cycles in \ensuremath{X}}\}=\ker(\partial_{p})$
\item $B_{p}(X)=\{\text{\ensuremath{p}-boundaries in \ensuremath{X}}\}=\text{Img}(\partial_{p+1})$.
\end{itemize}
Then $B_{p}(X)\leq Z_{p}(X)$ (subgroup) since $\partial\circ\partial=0$.
\begin{defn}
The $p$th singular homology group of $X$ is

\begin{align*}
H_{P}(X) & =\frac{Z_{p}(X)}{B_{p}(X)}\\
 & =\frac{\ker(\partial_{p})}{\text{Img}(\partial_{p+1})}.
\end{align*}


Geometrically, the elements of $H_{p}(X)$ are equivalence classes
of $p$-cycles where $\alpha\sim\alpha'$ if $\alpha'=\alpha+\partial\beta$,
$\beta\in C_{p+1}(X)$. We write $[\alpha]$ for the homology class
of a $p$-cycle $\alpha\in C_{p}(X)$.\end{defn}
\begin{lem}
If $\{X_{\alpha}\}$ are the path connected components of $X$, then
$C_{p}(X)=\bigoplus_{\alpha}C_{p}(X_{\alpha})$ and $\partial:C_{p}(X_{\alpha})\to C_{p-1}(X_{\alpha})$,
hence $H_{p}(X)=\bigoplus_{\alpha}H_{p}(X_{\alpha})$.\end{lem}
\begin{defn}
The singular chain complex of $X$

\begin{align*}
C_{p}\to C_{p-1}\to\dots\to C_{0}\to0
\end{align*}


is a sequence of Abelian groups and homomorphisms such that $\partial_{p-1}\circ\partial_{p}=0$.\end{defn}
\begin{thm}
(Interpretation of $H_{0}$)

If $X$ is non-empty and path connected, then $H_{0}(X)\cong\mathbb{Z}$.\end{thm}
\begin{cor}
If $X$ has $k$ path components, then $H_{0}(X)\cong\mathbb{Z}^{k}$.\end{cor}
\begin{thm}
(Interpretation of $H_{1}$) If $X$ is path connected, $x_{0}\in H$,
then $H_{1}(X)$ is the Abelianisation of $\pi_{1}(X,x_{0})$.
\end{thm}

\subsubsection{Effects of continuous maps}

If $f:X\to Y$ is continuous and $\sigma:\Delta^{p}\to X$ is a singular
$p$-simplex in $X$ then $f\circ\sigma:\Delta^{p}\to Y$ is a singular
$p$-simplex in $Y$. Then there exists an induced homomorphism $f_{\#}:C_{p}(X)\to C_{p}(Y)$
with $f_{\#}\left(\sum n_{i}\sigma_{i}\right)=\sum n_{i}(f\circ\sigma_{i})$.
\begin{lem}
$\partial\circ f_{\#}=f_{\#}\circ\partial$.\end{lem}
\begin{thm}
There are induced homomorphisms $f_{*}=H_{p}(f):H_{p}(X)\to H_{p}(Y)$,
where $f_{*}([c])=[f_{\#}c]$.\end{thm}
\begin{cor}
If $f:X\to Y$ is a homeomorphism, then $f_{*}:H_{p}(X)\to H_{p}(Y)$
is an isomorphism. Therefore $H_{p}$ are topological invariants.
\end{cor}

\subsubsection{Homotopy invariance of homology}
\begin{thm}
If $f,g$ are two homotopic maps $f,g:X\to Y$, then $f_{*}=g_{*}:H_{p}(X)\to H_{p}(Y),$
$p\geqslant0$.\end{thm}
\begin{cor}
If $f:X\to Y$ is a homotopy equivalence, then $f_{*}:H_{p}(X)\to H_{p}(Y)$
is an isomorphism.\end{cor}
\begin{defn}
(Simplicial homology)

A $\Delta$-complex is a topological space $X$ obtained from the
disjoint uninion of simplies, with ordered vertices glued together
by affine homeomorphisms.
\end{defn}
Define $C_{n}=C_{n}^{\Delta}(X)=\left\{ \sum n_{i}\sigma_{i}\mid n_{i}\in\mathbb{Z},\,\sigma_{i}\,\text{geometric \ensuremath{n}-simplices in \ensuremath{X}}\right\} $.
Then define $\partial:C_{n}\to C_{n-1}$ as before. Then we define
simplicial homology

\begin{align*}
H_{n}^{\Delta}(X) & =\frac{\ker(\partial_{n})}{\text{Img}(\partial_{n+1})}
\end{align*}

\begin{thm}
If $X$ is a $\Delta$-complex then $H_{n}^{\Delta}(X)\cong H_{n}(X)$.
(I.e. ``simplicial homology'' = ``singular homology'').\end{thm}
\begin{defn}
(Reduced homology) If $X=\emptyset$ then we can define an augmented
singular chain complex

\begin{align*}
\cdots\to C_{n}\to^{\partial_{n}}\partial_{n-1}\to^{\partial_{n-1}}\cdots\partial_{1}\to^{\partial_{1}}C_{0}\to^{\epsilon}\mathbb{Z}\to0
\end{align*}


where we let $\epsilon\left(\sum n_{x}x\right)=\sum n_{x}$. Now $\tilde{H}_{n}(X)=0$
if $X$ is contractible.
\end{defn}

\subsubsection{Exact sequences}
\begin{defn}
A sequence (finite or infinite) of Abelian groups and homomorphisms

\begin{align*}
\xymatrix{\cdots\ar[r] & A_{n}\ar[r]^{\alpha_{n}} & A_{n-1}\ar[r]^{\alpha_{n-1}} & \cdots}
\end{align*}


is called \emph{exact} if the image of each homomorphism is the kernal
of the next. I.e. $\text{Img}(\alpha_{n})=\ker(\alpha_{n-1})$ $\forall n$.
\end{defn}

\paragraph{Examples}
\begin{enumerate}
\item $0\to A\to^{f}B$ exact $\Leftrightarrow\ker(f)=\{0\}\Leftrightarrow f$
is 1-1 (injective)
\item $A\to^{f}B\to^{0}0$ exact $\Leftrightarrow\text{Img}(f)=\ker(B)=0$
$\Leftrightarrow f$ is onto (surjective).
\item $0\to^{0}A\to^{f}B\to^{0}0$ exact $\Leftrightarrow$ $f$ is an isomorphism.
\item $0\to^{0}A\to^{f}B\to^{g}C\to^{0}0$ exact $\Leftrightarrow$ $f$
is 1-1, $g$ is onto, $C=\text{Img}(g)\cong B/\ker(g)=B/\text{Img}(f)$\end{enumerate}
\begin{defn}
A short exact sequence $0\to A\to^{f}B\to^{g}C\to0$ splits if there
exists a homomorphism such that $g\circ h=1_{C}$ ($h:C\to B$). Then
$B=f(A)\oplus h(C)\cong A\oplus B$. The sequence splits if, for example,
$C$ is a free Abelian group.
\end{defn}



\paragraph{Mayer-Vietoris sequence}

An analogue to SvK. Let $X=X_{1}\cup X_{2}$ be a topological space,
where $X_{1},X_{2},X_{0}=X_{1}\cap X_{2}$ are open sets in $X$,
then we have inclusion maps

\begin{align*}
\xymatrix{ & X_{1}\ar[dr]_{j_{1}}\\
X_{0}\ar[ur]^{i_{1}}\ar[dr]_{i_{2}} &  & X\\
 & X_{2}\ar[ur]_{j_{2}}
}
\end{align*}


then we have

\begin{align*}
\xymatrix{ & H_{p}(X_{1})\ar[dr]_{j_{1*}}\\
H_{P}(X_{0})\ar[ur]^{i_{1*}}\ar[dr]_{i_{2*}} &  & H_{p}(X)\\
 & H_{p}(X_{2})\ar[ur]_{j_{2*}}
}
\end{align*}

\begin{thm}
(Mayer-Vietoris theorem) There is a long exact sequence in Homology

\begin{align*}
\xymatrix{\cdots\ar[r] & H_{p}(X_{0})\ar[r]^{i_{1*}\oplus i_{2*}} & H_{p}(X_{1})\oplus H_{p}(X_{2})\ar[r]^{j_{1*}-j_{2*}} & H_{p}(X)\ar[r]^{\partial_{*}} & H_{p-1}(X_{0})}
\end{align*}

\end{thm}


From this we can calculate the homotopy of the $n$-sphere as 

\begin{align*}
\tilde{H}_{p}(S^{n}) & =\begin{cases}
\mathbb{Z} & p=n\\
0 & \text{otherwise}
\end{cases}
\end{align*}

\begin{thm}
(Topological invariance of dimension) If $\mathbb{R}^{n}\approx\mathbb{R}^{m}$,
then $n=m$. \end{thm}
\begin{proof}
If $f:\mathbb{R}^{n}\to\mathbb{R}^{m}$ is a homeomorphism then we
also have an induced homeomorphism $\mathbb{R}^{n}-\{pt\}\to\mathbb{R}^{m}-\{pt\}$
so we have a homotopy $S^{n-1}\simeq S^{m-1}$ so $\tilde{H}_{*}(S^{n-1})\cong\tilde{H}_{*}(S^{m-1})$
so $n=m$.\end{proof}
\begin{defn}
A continous map $f:S^{n}\to S^{n}$ has degree $d$ if $f_{*}=H_{n}(f):\tilde{H}_{n}(S^{n})\to\tilde{H}_{n}(S^{n})$
is multiplication be the integer $d$.\end{defn}
\begin{rem}
$f\simeq g\implies\deg(f)=\deg(g)$. $\deg(f\circ g)=\deg(f)\times\deg(g)$.\end{rem}
\begin{thm}
(Subdivision theorem) Let $U=\{U_{j}\}$ be an open cover of $X$
(or a collection of subsets of $X$ whose inteiors cover $X$.). Then
let $C_{p}^{U}(X)$ be the subgroup of $C_{p}(X)$ consisting of singular
$p$-chains $\sum n_{i}\sigma_{i}$ such that the image $\sigma_{i}(\Delta^{p})$
of each $\sigma_{i}$ lies in one of the sets in $U$.

Then $\partial:C_{p}^{U}(X)\to C_{p-1}^{U}(X)$, so we get a chain
complex $C_{*}^{U}(X)$ with homology $H_{*}^{U}(X)$.

The inclusion $C_{*}^{U}(X)\to C_{*}(U)$ induces an isomorphism $H_{p}^{U}(X)\to H_{p}(X)$
for all $p$.\end{thm}
\begin{defn}
(Relative homotopy) Let $(X,A)$ be a pair of topological spaces with
$A\leq X$ ($A$ is a subspace.) The inclusion $A\hookrightarrow X$
induces the inclusions $C_{p}(A)\to C_{p}(X)$ for all $p$ and we
can define

\begin{align*}
C_{p}(X,A) & =\frac{C_{p}(X)}{C_{p}(A)}.
\end{align*}

\end{defn}
Then $\partial:C_{p}(X)\to C_{p-1}(X)$ takes $C_{p}(A)\to C_{p-1}(A)$
so induces a homomorphism $\bar{\partial}:C_{p}(X,A)\to C_{p-1}(X,A)$
and $\bar{\partial}\circ\bar{\partial}=0$. This gives a chain complex
with homology $H_{p}(X,A)=\frac{\ker(\bar{\partial}_{p})}{\text{Img}(\bar{\partial}_{p+1})}$
called the ``relative homology of $X$ mod $A$.''

